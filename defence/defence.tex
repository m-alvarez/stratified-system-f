\documentclass{beamer}
\usepackage[utf8]{inputenc}
\usepackage[T1]{fontenc}
%\usepackage[pdftex]{graphicx}
\usepackage[english]{babel}
\usepackage{amsmath}
\usepackage{datetime}
\usetheme{Warsaw}

\setbeamercolor{footline}{fg=black}

\addtobeamertemplate{navigation symbols}{}{%
    \usebeamerfont{footline}%
    \usebeamercolor[fg]{footline}%
    \hspace{1em}%
    \insertframenumber/\inserttotalframenumber
}

\title{Implementing Stratified System F in Coq}
\author{Mario Alvarez and Yannis Juglaret}
\newdate{date}{13}{3}{2015}
\date{\displaydate{date}}

\begin{document}

\begin{frame}
\titlepage
\end{frame}

%% - Summarize what you have formalized in this project. Recall briefly what the
%% project was about, the main steps of the formalization. Describe the part of the
%% project you addressed if you did not answer all the questions, describe your
%% personal work if you did more than what was required.

%% - Explain and comment your formalization choices, if applicable mention the
%% possible alternatives you investigated and the reasons of your final choice.

%% - If applicable, comment what was difficult/tedious, why, the possible
%% suggestion/solution you can see to these problems would you have more time to
%% work on it.

\section{Overview of our Formalization}

\subsection{Stratified System F}

\begin{frame}
  \frametitle{About Stratified System F}

%% TODO Present stratified system F.
\end{frame}

\begin{frame}
  \frametitle{Our Project}

  \begin{itemize}
    \item a formalization of a predicative System F ;
    \item correct type and kind inference algorithms ;
    \item proofs of meta-properties ;
    \item operational semantics ;
    \item . %% TODO Put something about strong normalization here.
  \end{itemize}

\end{frame}

\subsection{Definitions and Correctness Properties}

\begin{frame}[fragile]

  \frametitle{A Formalization of a Predicative System F}

\begin{verbatim}
Definition kind := nat.
Inductive typ := [...].
Inductive term := [...].
Inductive env := [...].

Inductive kinding : env -> typ -> kind -> Prop := [...].
Inductive typing : env -> term -> typ -> Prop := [...].
\end{verbatim}

\end{frame}

\begin{frame}[fragile]

\frametitle{Correct Type and Kind Inference Algorithms}

\begin{verbatim}
Fixpoint kind_of (e : env) (T : typ) : option kind := [...].
Fixpoint type_of (e : env) (t : term) : option typ := [...].

Theorem kind_of_correct (T : typ) :
  forall (e : env) (K : kind), 
  kind_of e T = Some K -> kinding e T K.

Theorem type_of_correct (t : term) : 
  forall (e : env) (T : typ), 
  type_of e t = Some T -> typing e t T.
\end{verbatim}

\end{frame}

\subsection{Metatheory}

\begin{frame}[fragile]

\frametitle{Proofs of Meta-Properties}

\begin{verbatim}
Theorem regularity :
  forall (e : env) (t : term) (T : typ),
    typing e t T ->
    exists (K : kind), kinding e T K.

Theorem narrowing :
  forall (e' : env) (t : term) (T : typ),
    typing e' t T ->
    forall (pos : nat) (e e'' : env) (K' K'' : kind),
    insert_kind pos K' e e' ->
    insert_kind pos K'' e e'' ->
    K'' <= K' ->
    typing e'' t T.
\end{verbatim}

\end{frame}

\subsection{Reduction and Strong Normalization}

\begin{frame}[fragile]

\frametitle{Operational Semantics}

\begin{verbatim}
Inductive reduction : term -> term -> Prop := [...].

Inductive normal : term -> Prop := [...]
  with neutral : term -> Prop := [...].

Theorem normality_preserved (t : term) (H : normal t) :
  forall (pos : nat) (T : typ), normal (subst_typ t pos T).

Theorem neutrality_preserved (t : term) (H : neutral t) :
  forall (pos : nat) (T : typ), neutral (subst_typ t pos T).
\end{verbatim}

%% TODO: add stuff from StrongNormalization.v (maybe on another slide)

\end{frame}

\section{Design Choices and Alternatives}

\subsection{Definitions}

\begin{frame}[fragile]

\frametitle{Variables and Environments}

\begin{itemize}
\item Type and kind variables: de Bruijn indices (distinct
  counters) ;
\item Environments:
  \begin{enumerate}
    \item \verb|list (typ + kind)| ;
    \item \verb|list typ * list kind| ;
  \end{enumerate}
\item 1 carries along the order of definitions ;
\item So choosing 1 makes well-formedness easy to state.
\end{itemize}

\begin{verbatim}
Fixpoint wf_env (e : env) : Prop :=
    match e with
      | empty      => True
      | evar T e'  => wf_typ e' T /\ wf_env e'
      | etvar K e' => wf_env e'
    end.
\end{verbatim}

\end{frame}

\section{Problems}

\begin{frame}
\end{frame}

%% No section for the final slide
\section*{}

\begin{frame}[fragile]
  \begin{center}
    Thank you for your attention!
  \end{center}
\end{frame}

\end{document}
