\documentclass{beamer}
\usepackage[utf8]{inputenc}
\usepackage[T1]{fontenc}
%\usepackage[pdftex]{graphicx}
\usepackage[english]{babel}
\usepackage{amsmath}
\usepackage{datetime}
\usepackage{verbments} % Requires verbments & Pygments
\usetheme{Warsaw}

\setbeamercolor{footline}{fg=black}
\plset{language=coq,fontsize=\small}

\addtobeamertemplate{navigation symbols}{}{%
    \usebeamerfont{footline}%
    \usebeamercolor[fg]{footline}%
    \hspace{1em}%
    \insertframenumber/\inserttotalframenumber
}

\title{Implementing Stratified System F in Coq}
\author{Mario Alvarez and Yannis Juglaret}
\newdate{date}{13}{3}{2015}
\date{\displaydate{date}}

\begin{document}

\begin{frame}
\titlepage
\end{frame}

%% - Summarize what you have formalized in this project. Recall briefly what the
%% project was about, the main steps of the formalization. Describe the part of the
%% project you addressed if you did not answer all the questions, describe your
%% personal work if you did more than what was required.

%% - Explain and comment your formalization choices, if applicable mention the
%% possible alternatives you investigated and the reasons of your final choice.

%% - If applicable, comment what was difficult/tedious, why, the possible
%% suggestion/solution you can see to these problems would you have more time to
%% work on it.

\section{Overview of our Formalization}

\subsection{Stratified System F}

\begin{frame}
  \frametitle{About Stratified System F}

%% TODO Present stratified system F.
\end{frame}

\begin{frame}
  \frametitle{Our Project}

  \begin{itemize}
    \item a formalization of a predicative System F ;
    \item correct type and kind inference algorithms ;
    \item proofs of meta-properties ;
    \item operational semantics ;
    \item . %% TODO Put something about strong normalization here.
  \end{itemize}

\end{frame}

\subsection{Definitions and Correctness Properties}

\begin{frame}[fragile]

  \frametitle{A Formalization of a Predicative System F}

\begin{pyglist}
Definition kind := nat.
Inductive typ := (* ... *).
Inductive term := (* ... *).
Inductive env := (* ... *).
\end{pyglist}

\begin{pyglist}
Inductive kinding : env -> typ -> kind -> Prop := (* ... *).
Inductive typing : env -> term -> typ -> Prop := (* ... *).
\end{pyglist}

\end{frame}

\begin{frame}[fragile]

\frametitle{Correct Type and Kind Inference Algorithms}

\begin{pyglist}
Fixpoint kind_of (e : env) (T : typ) :
  option kind := (* ... *).
\end{pyglist}

\begin{pyglist}
Fixpoint type_of (e : env) (t : term) :
  option typ := (* ... *).
\end{pyglist}

\begin{pyglist}
Theorem kind_of_correct (T : typ) :
  forall (e : env) (K : kind), 
  kind_of e T = Some K -> kinding e T K.
\end{pyglist}

\begin{pyglist}
Theorem type_of_correct (t : term) : 
  forall (e : env) (T : typ), 
  type_of e t = Some T -> typing e t T.
\end{pyglist}

\end{frame}

\subsection{Metatheory}

\begin{frame}[fragile]

\frametitle{Proofs of Meta-Properties}

\begin{pyglist}
Theorem regularity :
  forall (e : env) (t : term) (T : typ),
    typing e t T ->
    exists (K : kind), kinding e T K.
\end{pyglist}

\begin{pyglist}
Theorem narrowing :
  forall (e' : env) (t : term) (T : typ),
    typing e' t T ->
    forall (pos : nat) (e e'' : env) (K' K'' : kind),
    insert_kind pos K' e e' ->
    insert_kind pos K'' e e'' ->
    K'' <= K' ->
    typing e'' t T.
\end{pyglist}

\end{frame}

\subsection{Reduction and Strong Normalization}

\begin{frame}[fragile]

\frametitle{Operational Semantics}

\begin{pyglist}
Inductive reduction : term -> term -> Prop := (* ... *).
\end{pyglist}

\begin{pyglist}
Inductive normal : term -> Prop := (* ... *)
  with neutral : term -> Prop := (* ... *).
\end{pyglist}

\begin{pyglist}
Theorem normality_preserved (t : term) (H : normal t) :
  forall (pos : nat) (T : typ), normal (subst_typ t pos T).
\end{pyglist}

\begin{pyglist}
Theorem neutrality_preserved (t : term) (H : neutral t) :
  forall (pos : nat) (T : typ), neutral (subst_typ t pos T).
\end{pyglist}

%% TODO: add stuff from StrongNormalization.v (maybe on another slide)

\end{frame}

\section{Formalization Choices and Alternatives}

\subsection{Definitions}

\begin{frame}[fragile]

\frametitle{Variables and Environments}

\begin{itemize}
\item Type and kind variables: de Bruijn indices (distinct
  counters) ;
\item Environments:
  \begin{enumerate}
    \item \verb|list (typ + kind)| ;
    \item \verb|list typ * list kind| ;
  \end{enumerate}
\item 1 carries along the order of definitions ;
\item So choosing 1 makes well-formedness easy to state.
\end{itemize}

\begin{pyglist}
Fixpoint wf_env (e : env) : Prop :=
  match e with
    | empty      => True
    | evar T e'  => wf_typ e' T /\ wf_env e'
    | etvar K e' => wf_env e'
  end.
\end{pyglist}

\end{frame}

\subsection{Inductive Predicates for Deduction Rules}

\begin{frame}[fragile]

\frametitle{Let's Compare}

\begin{pyglist}[fontsize=\scriptsize]
Fixpoint kinding (e : env) (T : typ) (K : kind) : Prop :=
  match T with
  | tvar X => match get_kind e X with
              | None => False
              | Some K' => wf_env e /\ K' <= K
              end
  | tarr T1 T2 => exists K1 K2 : kind,
      max K1 K2 = K /\ kinding e T1 K1 /\ kinding e T2 K2
  | tall K1 T1 => exists K' : kind,
      kinding (etvar K1 e) T1 K' /\ K = S (max K1 K')
  end.
\end{pyglist}

\begin{pyglist}[fontsize=\scriptsize]
Inductive kinding : env -> typ -> kind -> Prop :=
| k_tvar (e : env) (X : nat) (Kp Kq : kind) :
  wf_env e -> get_kind e X = Some Kp -> Kp <= Kq -> kinding e (tvar X) Kq
| k_tall (e : env) (T : typ) (Kp Kq : kind) :
  kinding (etvar Kq e) T Kp -> kinding e (tall Kq T) (S (max Kp Kq))
| k_tarr (e : env) (T1 T2 : typ) (Kp Kq : kind) :
  kinding e T1 Kp -> kinding e T2 Kq -> kinding e (tarr T1 T2) (max Kp Kq)
.
\end{pyglist}

\end{frame}

\begin{frame}[fragile]

\frametitle{You Don't Want This…}

\begin{pyglist}[fontsize=\scriptsize]
Fixpoint typing (e : env) (t : term) (T : typ) : Prop :=
  match t with
  | var x => match get_typ e x with
             | None => False
             | Some T' => wf_env e /\ T = T'
             end
  | abs T' t' => match T with
                 | tarr T1 T2 =>
                     T1 = T' /\ typing (evar T1 e) t' T2
                 | _ => False
                 end
  | app t1 t2 => exists T2 : typ, typing e t1 (tarr T2 T) /\ typing e t2 T2
  | abs_t k t' => match T with
                  | tall k' T' =>
                      k' = k /\ typing (etvar k e) t' T'
                  | _ => False
                  end
  | app_t t' T' => exists l : kind, exists T1 : typ,
      typing e t' (tall l T1) /\ kinding e T' l /\ T = tsubst T1 0 T'
  end.
\end{pyglist}

\end{frame}

\begin{frame}[fragile]

\frametitle{You Want This!}

\begin{itemize}
  \item Clearer for the programmer.
  \item Much easier proofs.
\end{itemize}

\begin{pyglist}[fontsize=\scriptsize]
Inductive typing : env -> term -> typ -> Prop :=
| t_var (e : env) (X : nat) (T : typ) :
    wf_env e -> get_typ e X = Some T ->
    typing e (var X) T
| t_abs (e : env) (T1 T2 : typ) (t : term) :
    wf_typ e T1 -> typing (evar T1 e) t T2 ->
    typing e (abs T1 t) (tarr T1 T2)
| t_app (e : env) (T1 T2 : typ) (t1 : term) (t2 : term) :
    typing e t1 (tarr T1 T2) -> typing e t2 T1 ->
    typing e (app t1 t2) T2
| t_abs_t (e : env) (K : kind) (t : term) (T : typ) :
    typing (etvar K e) t T -> typing e (abs_t K t) (tall K T)
| t_app_t (e : env) (K : kind) (t : term) (T1 T2 : typ) :
    typing e t (tall K T1) -> kinding e T2 K ->
    typing e (app_t t T2) (tsubst T1 0 T2)
.
\end{pyglist}

\end{frame}

\section{Problems}

\begin{frame}[fragile]
\frametitle{A Wrong Definition for Kind Insertion}

\begin{pyglist}[fontsize=\scriptsize]
Definition insert_kind (v : nat) := insert_kind_r v v.
Fixpoint insert_kind_r (i : nat) (v : nat) (e : env) (e' : env) : Prop :=
match e with
| evar t e1 => match e' with
               | evar t' e1' =>
                   (tshift v t) = t' /\ insert_kind_r i v e1 e1'
               | _ => False
               end
| etvar k e1 => match e' with
                | etvar k' e1' => match i with
                                  | 0 => etvar k e1 = e1'
                                  | S i1 =>
                                      k = k' /\ insert_kind_r i1 v e1 e1'
                                  end
                | _ => False
                end
  | empty => match e' with
             | etvar k empty => v = 0
             | _ => False
             end
end.
\end{pyglist}

\end{frame}

\begin{frame}[fragile]
\frametitle{Our Final Definition for Kind Insertion}

\begin{pyglist}
Inductive insert_kind : nat -> kind -> env -> env -> Prop :=
| ik_top K e :
    insert_kind 0 K e (etvar K e)
| ik_evar pos K T e e' :
    insert_kind pos K e e' ->
    insert_kind pos K (evar T e) (evar (tshift pos T) e')
| ik_etvar pos K K' e e' :
    insert_kind pos K e e' ->
    insert_kind (S pos) K (etvar K' e) (etvar K' e')
.
\end{pyglist}

\end{frame}

%% No section for the final slide
\section*{}

\begin{frame}[fragile]
  \begin{center}
    Thank you for your attention!
  \end{center}
\end{frame}

\end{document}
