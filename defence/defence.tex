\documentclass{beamer}
\usepackage[utf8]{inputenc}
\usepackage[T1]{fontenc}
%\usepackage[pdftex]{graphicx}
\usepackage[english]{babel}
\usepackage{amsmath}
\usepackage{datetime}
\usetheme{Warsaw}

\setbeamercolor{footline}{fg=black}

\addtobeamertemplate{navigation symbols}{}{%
    \usebeamerfont{footline}%
    \usebeamercolor[fg]{footline}%
    \hspace{1em}%
    \insertframenumber/\inserttotalframenumber
}

\title{Implementing Stratified System F in Coq}
\author{Mario Alvarez and Yannis Juglaret}
\newdate{date}{13}{3}{2015}
\date{\displaydate{date}}

\begin{document}

\begin{frame}
\titlepage
\end{frame}

%% - Summarize what you have formalized in this project. Recall briefly what the
%% project was about, the main steps of the formalization. Describe the part of the
%% project you addressed if you did not answer all the questions, describe your
%% personal work if you did more than what was required.

%% - Explain and comment your formalization choices, if applicable mention the
%% possible alternatives you investigated and the reasons of your final choice.

%% - If applicable, comment what was difficult/tedious, why, the possible
%% suggestion/solution you can see to these problems would you have more time to
%% work on it.

\section{Overview of our Formalization}

\subsection{Stratified System F}

\begin{frame}
%% TODO Present stratified system F.
\end{frame}

\begin{frame}
  In this project, we have :

  \begin{itemize}
    \item defined our formalization of a predicative System F ;
    \item defined correct type and kind inference algorithms ;
    \item stated and proved meta-properties about our language ;
    \item defined an operational semantics for our system ;
    \item . %% TODO Put something about strong normalization here.
  \end{itemize}
\end{frame}

\subsection{Definitions and Correctness Properties}

\begin{frame}[fragile]

\begin{verbatim}
Definition kind := nat.
Inductive typ := [...].
Inductive term := [...].
Inductive env := [...].

Inductive kinding : env -> typ -> kind -> Prop := [...].
Inductive typing : env -> term -> typ -> Prop := [...].
\end{verbatim}

\end{frame}

\begin{frame}[fragile]

%% This text overlaps

%% We defined type and kind inference algorithms and proved them correct.

\begin{verbatim}
(** [kind_of e T] provides a minimal kind for type [T] in
    environment [e], if [T] is kindable in [e]. *)
Fixpoint kind_of (e : env) (T : typ) : option kind := [...].

(** [type_of e t] provides a type for term [t] in environment [e],
    if [t] is typable in [e]. *)
Fixpoint type_of (e : env) (t : term) : option typ := [...].

Theorem kind_of_correct (T : typ) :
  forall (e : env) (K : kind), 
  kind_of e T = Some K -> kinding e T K.

Theorem type_of_correct (t : term) : 
  forall (e : env) (T : typ), 
  type_of e t = Some T -> typing e t T.
\end{verbatim}

\end{frame}

\subsection{Metatheory}

\begin{frame}[fragile]

\begin{verbatim}
Theorem regularity :
  forall (e : env) (t : term) (T : typ),
    typing e t T ->
    exists (K : kind), kinding e T K.

Theorem narrowing :
  forall (e' : env) (t : term) (T : typ),
    typing e' t T ->
    forall (pos : nat) (e e'' : env) (K' K'' : kind),
    insert_kind pos K' e e' ->
    insert_kind pos K'' e e'' ->
    K'' <= K' ->
    typing e'' t T.
\end{verbatim}

\end{frame}

\subsection{Reduction and Strong Normalization}

\begin{frame}[fragile]

\begin{verbatim}
Inductive reduction : term -> term -> Prop := [...].

Inductive normal : term -> Prop := [...]
  with neutral : term -> Prop := [...].

Theorem normality_preserved (t : term) (H : normal t) :
  forall (pos : nat) (T : typ), normal (subst_typ t pos T).

Theorem neutrality_preserved (t : term) (H : neutral t) :
  forall (pos : nat) (T : typ), neutral (subst_typ t pos T).
\end{verbatim}

%% TODO: add stuff from StrongNormalization.v

\end{frame}

%\section{}

%\section{}

\begin{frame}[fragile]
  \begin{center}
    Thank you for your attention!
  \end{center}
\end{frame}

\end{document}
