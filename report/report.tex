\documentclass[a4paper,11pt]{article}

%% ===== Instructions reminder =====

%% - The main purpose of this project is to evaluate your approach to
%% formalization, and your comments on the value of the solutions you
%% tried. Investigating several possible definitions/proof methods is a
%% good thing and we value your description of the problems encountered
%% and of the solutions you eventually preferred as much as the the
%% number of questions you answer too. The best mark will not necessarily
%% go to the project answering the maximum number of questions.

%% - Do not hesitate to insert comments in your code...

%% - The report should be a separate, readable document, ideally in pdf
%% format, written in English or in French. It should not exceed 5 pages
%% of a reasonably readable document style (no magnifying glass
%% needed). It can refer to the code but should explain the approach, the
%% problems encountered, and comment the code submitted: is it
%% satisfactory (data-structures, statements, proof scripts) or
%% were some proofs more difficult (or easier!) formally than on paper,
%% what was difficult and why, what ingredient or what method worked
%% well, did you use/lack of some automation tools...

%% ===== Basis for the report =====

%% - First, an important note: the file structure of the code.
%% - The choice of indexing variables: types and kinds are numbered
%%   separately, etcetera. Why we chose it.
%% - We started defining typing and kinding as Fixpoints instead of
%%   Inductive predicates. This made everything really hard!
%% - Equality for types: it's not enough to have =, we needed beq_typ
%%   because you can't use = in if...then...else... (and this is necessary for
%%   type_of, for example). This means we also needed some lemmas (like the ones
%%   in the exam!) showing correctness (but not completeness!) for this
%%   equality, beq_typ
%% - We got the wrong definition for insert_kind. This was extremely
%%   painful, because we got it wrong in subtle ways so many theorems were still
%%   true, but not all, and it took us a while to realize that it was wrong. You
%%   should probably check the Git logs to see the previous version. The proofs
%%   were HUGE and very different. This is very interesting: things like "does
%%   this type have a kind?" were based in "*how many* free variables does it
%%   have?" and "*how many* kinds are in the environment?" This is
%%   interesting. The new way of doing this is through some "extensionality"
%%   lemmas that say that "the kind of a type only depends on the kinds in the
%%   environment". Take a look at these and understand how powerful they are!
%%   Same for typing, but typing extensionality is more complex! Again, read the
%%   code, understand the lemmas. Maybe it's interesting if you go backwards:
%%   look at the final theorems (the ones that are required by the exercise) and
%%   see how I defined lemmas in order to prove the required stuff!
%% - Reduction.v is very interesting because I started doing advanced
%%   stuff: you have examples of notations and tactics. It's worth mentioning
%%   the tactics (they are easy to understand) and how they make the congruence
%%   lemmas very easy to prove.
%% - Also, please, if you could format the files nicely and group related
%%   stuff together and put comments, that would be nice. If not, just add a
%%   header to each file saying "this file contains this and that, for this
%%   exercise".
%% - StrongNormalization.v is incomplete...and is never going to be
%%   completed. The incomplete code is in the comments.
%% - Some cool tactics and stuff:
%% > - *eauto, eapply, erewrite* - they usually avoid uses of 'specialize'
%% > - *pose* - good when you want to take a lemma from the outside and put
%% >   it in your proof
%% > - *auto/eauto with arith, auto/eauto using [lemma_1, lemma_2...]* -
%% >   good for when the proof is trivial but you need to remember a few lemmas
%% >   that are defined outside. It avoids you having to pose the lemmas.
%% > - The tactic language: *destruct; someTactic.* will apply someTactic
%% >   to all branches of the destruct/case/induction in parallel! This saves a
%% >   lot of inductions followed by intros in every branch or stuff like that.
%% > - *try* allows you to apply a tactic and, if it doesn't work, it's ok.
%% >   this is good because some tactics like omega or discriminate will fail if
%% >   they don't work! you can't just tell coq "use omega everywhere and only ask
%% >   me to prove the cases where it doesn't work", because if it doesn't work it
%% >   will raise an error. Instead, "use *try omega *everywhere". Instead of
%% >   failing, it will let you continue the proof.
%% > - *discriminate/try discriminate* - it's good when you have a case
%% >   analysis but some cases are impossible - like if you know some term CAN'T
%% >   be a variable. Discriminate will take care of those cases.
%% >
%% >

% We may want to use this if we get more than 5 pages
%\usepackage[a4paper,includeheadfoot,margin=2.4cm,footskip=.5cm]{geometry}

\usepackage[utf8]{inputenc}
\usepackage[T1]{fontenc}
\usepackage[pdftex]{graphicx}
\usepackage[english]{babel}
%\usepackage{amsmath}
%\usepackage{amsthm}
%\usepackage{amsfonts}

\begin{document}
\title{Implementing Stratified System F in Coq}
\author{M. Alvarez and Y. Juglaret}
\date{}
\maketitle

In this project, we formalized a version of predicative System F and
proved interesting properties about it, mostly based on the paper
\emph{Hereditary Substitution for Stratified System F} by H. D. Eades
III and A. Stump.

\section{Architecture of the Project}

As the script file grew larger and larger, we decided to split it into
several distinct files that group altogether related code. We thus
provide :

\begin{itemize}
  \item \verb|SysF.v| and \verb|Correctness.v| for part \emph{I.1.2
    Definitions} ;
  \item \verb|Metatheory.v| for part \emph{I.1.3 Basic Metatheory} ;
  \item \verb|Reduction.v| for part \emph{I.2 Reduction and Normal
    Terms} ;
  \item \verb|StrongNormalization.v| as a basis for the optional
    part \emph{II.1 Strong Normalization}.
\end{itemize}

We provide compilation rules for each distinct file, which helps
for sanity check. That is, running \verb|make| will compile all files
respecting the dependencies order, and thus fail if anything is wrong
within the whole project.

Since every other file depends on the definitions in \verb|SysF.v|,
using any of these files requires having compiled \verb|SysF.v|
first. This may be done using \verb|make SysF.vo|. Also,
\verb|StrongNormalization.v| relies on the whole definitions, it is
thus recommended to have run \verb|make| once before trying to execute
it.

\subsection{Definitions}

[...]

We also define a structural decidable egality predicate over types
\verb|beq_typ : typ -> typ -> bool|. We need it for example in
\verb|type_of : env -> term -> option typ|, where we need to test for
equality in a conditional statement. This can't be done using
the usual equality \verb|eq|, whose result is in \verb|Prop|.

Of course, like in the exam with the structural decidable equality over
Polynomials, this calls for the addition of some lemmas talking about
\verb|beq_typ| if we want to prove any interesting property about a
function that uses it. In particular, in \verb|Correctness.v| we
define two lemma that state that \verb|beq_typ| is correct and
reflexive.

\subsection{Basic Metatheory}

[...]

\subsection{Reduction and Normal Terms}

[...]

\subsection{Strong Normalization}

[...]


\section{Design Choices}

The definition part allows some liberty in the design choices. In this
section, we explain our choices and the reasons why we made them.

\subsection{Indexes for Kind and Type Variables}

We chose to index variables by separating types and kind variables.

[...]

\subsection{Kinding and Typing predicates}

On paper, the typing and kinding predicates are defined by deductive
rules. In our early implementations, we described them as recursive
predicates using \verb|Fixpoint| declarations. This was a very bad
idea, and proving anything related to these predicates required a lot
of efforts.

After a while, we learned about inductive predicates and how
appropriate they are for representing deductive rules in Coq. We thus
switched to \verb|Inductive| definitions for both, and thanks to the
inductive reasoning all properties became a lot easier to prove. Even
though this change implied having to adapt all of the proofs that we
had already done, in the end we have probably still gained time thanks
to it.

\end{document}
